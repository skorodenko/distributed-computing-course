% ******** Приклад оформлення документа за ДСТУ 3008-95 ********
% ******************** автор: Тавров Д. Ю. *********************

% зазначаємо стильовий файл, який будемо використовувати
\documentclass{udstu}

% починаємо верстку документа
\begin{document}

% створимо титульний аркуш
% за допомогою спеціальної команди
% \maketitlepage{params},
% де params --- це розділені комами пари "параметр={значення}"
\maketitlepage{
% StudentName --- прізвище, ініціали студента
	StudentName={Скорденко Д. О.},
% StudentMale --- стать студента (true, якщо чоловік, false --- якщо жінка)
	StudentMale=true,
% StudentGroup --- група студента
	StudentGroup={КМ-01},
% Title --- назва
	Title={Звіт\\із лабораторної роботи №1\\із дисципліни \invcommas{Розподілені і хмарні обчислення}},
% SupervisorDegree --- науковий ступінь, учене звання керівника роботи
% якщо наукового ступеня немає, можна відповідний рядочок пропустити
	SupervisorDegree={доцент кафедри ПМА},
% SupervisorName --- прізвище, ініціали керівника роботи
	SupervisorName={Ліскін В. О.}
}

% створюємо зміст
\tableofcontents

% створюємо вступ
\intro

\paragraph{\textbf{Мета:}} навчитись працювати із примітивом thread (потоком)

% створюємо перший розділ роботи
\chapter{Основна частина}
\label{chap:1}

\paragraph{\textbf{Опис програми:}}

Для демонстрації роботи із потоками буде реалізовано програму інкремент/декримент.
Кожна із операцій буде виконуватись в окремому потоці.


\chapter{Опис програми [Тестовий приклад]}
\label{chap:3}

Всі результати наводять роботу програми протягом 1-2 секунд.

\begin{figure}[!htp]
	\centering
	\includegraphics[scale=0.5]{PNG/no-prior.png}
	\caption{Відсутній пріорітет}
	\label{fig:figure1}
\end{figure}

\begin{figure}[!htp]
	\centering
	\includegraphics[scale=0.5]{PNG/min-prior-increment.png}
	\caption{Мінімальний пріорітет інкременту}
	\label{fig:figure1}
\end{figure}

\begin{figure}[!htp]
	\centering
	\includegraphics[scale=0.5]{PNG/min-prior-decrement.png}
	\caption{Мінімальний пріорітет декременту}
	\label{fig:figure1}
\end{figure}

% створюємо Висновки
\conclusions

Було реалізовано програму інкремент/декремент для демонстрації роботи потоків та їх пріоритезації.

\append{Код лістінги}

\paragraph{\textsc{*Примітка:}}
У код лістингах при копіюванні втрачається форматування (не копіюються пробіли).
Файли прикріплено до цього pdf (вкладка "прикріплені файли").

\captionof{listing}{main.rs}
\embedfile[filespec=main.rs]{../src/main.rs}
\inputminted{rust}{../src/main.rs}

\end{document}
